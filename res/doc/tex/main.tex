
\documentclass[oneside]{ausarbeitung}
\bibliography{latexlit}


% ----------------------------------------------------------------------

\begin{document}

%--- Sprachauswahl
% Erlaubte Werte:
%   \selectlanguage{english}
%   \selectlanguage{ngerman}
\selectlanguage{ngerman}

%--- Art der Arbeit
% Erlaubte Werte:
%   \Praxissemesterbericht
%   \Projektbericht
%   \Bachelorarbeit
%   \Seminararbeit
%   \Masterarbeit

\Projektbericht

%--- Studiengang:
% Erlaubte Werte:
%   \Informatik
%   \Elektronik
%   \DataScience
\MLD

\title{Titel der Arbeit}

\author{Nils Ostertag}
\matrikelnr{75217}

%--- Ist der Erstbetreuer (\examinerA) an der Hochschule ein Professor?
% Erlaubte Werte:
%   \examinerIsAProfessortrue   % Ja
\examinerIsAProfessorfalse  % Nein
%   \examinerIsAProfessortrue   % Ja

%--- Betreuer
\examinerA{Dr.~Marc~Hermann}
%\examinerB{Prof.~Dr.~Ulrich~Klauck}

%--- Einreichungsdatum
\date{15. August 2024}

%--- Angaben zur Firma
% Auskommentieren, wenn die Arbeit nicht bei einer ext. Firma gemacht wurde.
%\companyname{Beispielfirma}
%\industrialsector{Beispielbranche}
%\department{Beispielabteilung}
%\companystreet{Beispielstr. 1}
%\companycity{12345 Musterstadt}

%--- Angaben zum Betreuer bei dieser Firma
%\advisorname{Name des Betreuers}
%\advisorphone{(01234) 567-890}
%\advisoremail{name@company.xxx}

%--- Titelseite Anzeigen
\maketitle
\cleardoublepage

%---
\pagenumbering{roman}
\setcounter{page}{1}

%--- Firmendaten Anzeigen
% Auskommentieren, wenn die Arbeit nicht bei einer ext. Firma gemacht wurde.
%\makeworkplace
%\cleardoublepage

%--- Eidesstattliche Erklärung anzeigen
\makeaffirmation
\cleardoublepage

%---
\begin{abstract}
  Ziel der Kurzfassung ist es, einen (eiligen) Leser zu informieren, so 
  dass dieser entscheiden kann, ob der Bericht für ihn hilfreich ist oder 
  nicht (neudeutsch: Management Summary). Die Kurzfassung gibt daher eine 
  kurze Darstellung

  \begin{itemize}
    \item des in der Arbeit angegangenen Problems
    \item der verwendeten Methode(n)
    \item des in der Arbeit erzielten Fortschritts.
  \end{itemize}

  Dabei sollte nicht auf die Struktur der Arbeit eingegangen werden, also 
  Kapitel~\ref{cha:grundlagen} etc. denn die Kurzfassung soll ja gerade 
  das Wichtigste der Arbeit vermitteln, ohne dass diese gelesen werden muss.
  Eine Kapitelbezogene Darstellung sollte sich in Kapitel~%
  \ref{cha:einleitung} unter Vorgehen befinden.

  Länge: Maximal 1 Seite.
\end{abstract}
%-----------------------------------------------------------------------
\cleardoublepage
\tableofcontents

%---
\listoffigures

%---
\listoftables

%---
\lstlistoflistings

%---
\listofabbreviations
\begin{acronym}[Bsp.]  % Längstes Kürzel in der nachfolgenden
                       % Liste um die Breite der Spalte für die
                       % Abkürzungen zu bestimmen.

%% Eintrag: \acro{Referenzname}[Kürzel]{Langform}
%% Im Text wird die Abkürzung dann mit \ac{Referenzname} benutzt.
\acro{rup}[RUP]{Rational Unified Process}
%\acro{bsp}[Bsp.]{Beispiel}
\end{acronym}
%---


\cleardoublepage
\pagenumbering{arabic}
\setcounter{page}{1}

% ----------------------------------------------------------------------
\chapter{Einleitung}
\label{cha:einleitung}

Die Einleitung dient dazu, beim Leser Interesse für die Inhalte 
Praxissemesterberichts zu wecken, die behandelten Probleme aufzuzeigen 
und die zu ihrer Lösung entwickelten Konzepte zu beschreiben.

\section{Motivation}
\label{sec:motivation}

In der Motivation wird dargestellt, welche Bedeutung die im 
Praxissemester zu entwickelnden Lösungen für das betreuende Unternehmen 
haben. Es wird beispielsweise aufzeigt, in welches Produkt sie eingehen, 
welcher Ablauf verbessert werden soll etc.

\section{Problemstellung und -abgrenzung}
\label{sec:problemstellung}

Die Problemstellung dient dazu, das zu lösende Problem klar zu 
definieren und abzugrenzen. Der Praktikant soll ein klares Verständnis 
des zu lösenden Problems haben. Insbesondere soll auch verhindert 
werden, dass zu viele Probleme gleichzeitig angegangen werden. Eine 
Negativabgrenzung verhindert, dass beim Leser später nicht erfüllte 
Erwartungen geweckt werden.

\section{Ziel der Arbeit}
\label{sec:ziel}

Mit dem Ziel der Arbeit wird der angestrebte Lösungsumfang festgelegt. An diesem Ziel wird entschieden, ob das Praktikum erfolgreich absolviert wurde oder nicht.

\section{Vorgehen}
\label{sec:vorgehen}

Nachdem mit Problemstellung und Ziel gewissermaßen Anfangs- und Endpunkt 
des Praktikums beschrieben sind, wird hier der zur Erreichung des Ziels 
eingeschlagene Weg vorgestellt. Dazu werden typischerweise die folgenden 
Kapitel und ihr Beitrag zur Erreichung des Ziels der Arbeit kurz 
beschrieben. Die folgenden Kapitel sind ein – möglicher – Aufbau, 
Abweichungen können durchaus notwendig sein. Zur Darstellung des 
Vorgehens ist eine grafische Darstellung sinnvoll, bei der die einzelnen 
Lösungsschritte und ihr Zusammenhang dargestellt werden. Ein Beispiel 
hierfür findet sich in Abbildung \ref{fig:1}.

% ---

\chapter{Aktueller Forschungsstand}
\label[cha:forschungsstand]

\section{Predictive Maintenance}
\label{sec:research_predictivemaintenance}

\section{Fahrverhaltensanalyse}
\label{sec:research_fahrverhalten}


% ---
\chapter{Theoretische Grundlagen}
\label{cha:grundlagen}

In diesem Kapitel das für das Praktikum relevante Grundlagenwissen 
dargestellt. Der Praktikant soll hierzu das ihm durch Vorlesungen 
bekannte, bzw. durch Recherchen vertiefte theoretische Wissen 
darstellen, das für die Lösung der im Praktikum gestellten Probleme 
notwendig ist.

Dabei ist darauf zu achten, nur solche Inhalte in das Grundlagenkapitel 
aufzunehmen, die später auch verwendet werden (Problembezogenheit). 
Ebenso ist auf eine ausreichend tiefe und vollständige Darstellung der 
Grundlagen zu achten.

Für die Erstellung des Literaturverzeichnisses 
wird das Werkzeug JabRef\autocite{JabRef:JabRef} verwendet. 

Sie können aber auch das Werkzeug Citavi\autocite{SAS:Citavi} benutzen
und dort nach \textsc{Bib}\TeX{} exportieren.

\section{OBD-II}
\label{sec:foundations_obd2}

\section{Predictive Maintenance}
\label{sec:foundations_predictivemaintenance}

\section{Fahrverhalten und -Charakteristiken}
\label{sec:foundations_fahrverhalten}

\section{Machine Learning und Klassifizierung}
\label{sec:foundations_ml}

\section{Raspberry Pi}
\label{sec:foundations_raspberry}

\section{Node-RED}
\label{sec:foundations_nodered}

\section{WireGuard VPN}
\label{sec:foundations_wireguard}

\section{App-Development mit Flutter}
\label{sec:foundations_flutter}

%---
\chapter{Umsetzung und Implementierung}
\label{cha:implement}

\section{Anforderungsanalyse}
\label{sec:implement_requirements}

\section{Komponenten- und Systemübersicht}
\label{sec:implement_components}

\section{Data Aquisition / Subkomponente Mobile-App}
\label{sec:implement_dataaquisition}

\section{Backend}
\label{sec:implement_backend}

\section{Datengrundlage und Datenvorverarbeitung}
\label{sec:implement_dataprocessing}

\section{}

%---
\chapter{Evaluation}
\label{cha:evaluation}

%---
\chapter{Zusammenfassung und Ausblick}
\label{cha:zusammenfassung}

\section{Erreichte Ergebnisse}
\label{sec:ergebnisse}

\section{Fazit und Ausblick}
\label{sec:ausblick}

%-----------------------------------------------------------------------
\appendix

%---
\printbibliography[heading=bibintoc]

%---
\chapter{Anhang A}

%---
\chapter{Anhang B}


\end{document}